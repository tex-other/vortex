\section{Communications Protocols}

Along each edge of the process triangle, a separate communications
protocol has been defined.  The source editor-formatter and
source editor-proof editor connections are described in detail here.
Each of the three protocols include a global protocol, which
implements rendezvous and common functionality.

All numbers are transmitted in network byte order.  Two sizes are
used: {\bf long} (32 bytes) and {\bf short} (16 bytes) either signed
or unsigned.  Other data ({\ASCII} strings) is sent as a byte stream.
There are no padding requirements for messages, although we have been
careful to lay out the data within each message to avoid architecture
differences so that the data can be easily read into a C structure or
array.  Strings are not necessarily terminated with a NUL character
({\ASCII} 0).

Each packet begins with a constant format header.  The eight bytes of
header may make up the entire message.  These headers are constant,
however the trailing data is defined by each request.

\begin{tabbing}
\hskip.5in\=\bf u\_short \hskip.5in\= request \hskip.5in\= the request code\\
    \>\bf u\_short \> datalen \> the trailing data length (bytes)\\
    \>\bf u\_long \> commdata \> communications specific data\\
\end{tabbing}

The ``communications specific data'' is not used in the global
protocol, but stores the window~ID and file~ID for the proof editor
and formatter communications respectively.  These IDs allow the two
processes to pass the most common piece of information without
needing to define additional protocol.

\subsection{Global Protocol}

The basic format of all messages and the messages common to all three
edges of the triangle are listed in the include file {\tt "gl\_comm.h"}.
A copy of this file and the other three message sets are appended to
this report.  See those files for a more concise description of
the byte layout of the individual messages.

\subsubsection{Process Rendezvous}

Each of the two daemon processes needs to connect to the source editor
and to each other.  Upon start-up, each daemon examines its argument
list to determine how to call the source editor back.  Two non-option
arguments are expected, the host name and internet port number on which
the source editor is listening.  In addition, the proof editor takes the
X~display name as a third argument.  (Both programs should also allow
options to be passed in, particularly the {\tt -d} option for debugging.)

Once a daemon has successfully connected, it should send a {\tt VERIFY}
request on the new connection.  The source editor will respond with a
{\tt WELCOME} or {\tt GOAWAY}.  The {\tt VERIFY} packet contains an
identifier (a ``magic number'') and the protocol versions of the global
and specific local communications.  Assuming the identifier is correct
and the versions match, a {\tt WELCOME} is sent with no data, otherwise
it responds with a {\tt GOAWAY} message and the data contains an error
message (text string).  Once the {\tt WELCOME} message has been received,
the daemon is officially connected to the source editor and no more
hand-shaking is necessary.

\subsubsection{Connecting the Formatter and Proof Editor}

There is one other rendezvous task that must be done; connecting the
formatter and proof editor to one another.  The protocol we've implemented
to perform this is necessary because of the possibility of running more
than one {\VorTeX} on a single machine; we could not just assign
the internet port numbers.

The protocol used for this rendezvous involves a three step process:
1) the source editor instructs one daemon to listen for a connection
from the other with a {\tt LISTENAT} request, 2) the daemon responds
with a {\tt LISTENING} reply when it is ready, and 3) the source editor
instructs the other daemon to connect to the first one with a
{\tt CONNECT} request.

The {\tt LISTENAT} request includes an internet port number and a
time-out value.  However, the port number it specifies need not be
the one used by the daemon.  It must attempt to bind internet ports
starting with the number specified and incrementing until the operation
succeeds (or fails with some error other than ``port already in use'').
Once it has successfully bound the port, it returns the port number to
the source editor and waits for at most the given time-out for a
connection from the other daemon.

Once the two daemons have established the TCP/IP connection, all of
the system is in proper communication.  Whenever the second of the two
daemons successfully connects back to it, the source editor attempts
to establish a connection between the daemons as described
above.\footnote{The Lisp programmer may also explicitly establish the
connection with {\tt make-connection}.}

\subsubsection{Error Handling}

There are two main jobs involved in error handling; reporting and
re-synchronization.  The first is obvious, but the second requires
some throughout.  Since we've implemented the {\VorTeX} protocol on
TCP/IP streams (which have no message boundaries), we need some method
of marking the stream for re-synchronization.

This resynchronization is done with the ``out of band data'' facility
of {\UNIX} sockets.  The {\tt FLUSH} request sets a mark (using an
out of band character) and the receiver flushes all bytes up to that
mark.  In practice it has not happened that the programs lost their
position in the byte stream (since the headers are regular), but it
may be necessary at some time in the future.

There is an additional request, {\tt ERROR}, which sends an {\ASCII}
error message destined for the user.  These requests are really only
used to send errors back to the user to be displayed by the source
editor.  {\tt ERROR} is a request just like any other and does not
imply {\tt FLUSH} or {\tt ABORT}.

There are two messages which terminate the connections (and the daemon
processes when sent from the source editor).  {\tt QUIT} signals
termination (usually due to a user command) of a daemon and {\tt ABORT}
is considered an error termination.  Both have the same semantics
(no trailing data) and are handled the same way by the current source
editor.  When sent from the source editor, they command the specified
daemon to quit and when sent from a daemon, signal that the daemon is
quitting.  The daemons cannot command the source editor to terminate.

\subsection{Proof/Source Protocol}

The Lisp functions which implement proof editor functionality are all
executed by the source editor and presumably result in calls to the
lowest-level proof operations.  In addition to these, there are many
implicit operations which are not visible to the user such as {\it
create window\/}, {\em expose window\/} and {\em destroy
window}.\footnote{These implicit requests mirror the X version 10
window management paradigm.} See the include source file {\tt
"ps\_comm.h"} for the exact format of all the proof editor procedure
calls.

In all proof/source requests which require one, the {\em window ID} of
the proof window in question is passed in the {\tt commdata} field of
the message header (bytes 4 through 8).  The ID passed is actually the
X window~ID of the proof window (which corresponds to a proof window
and buffer for the source editor).  For all requests other than
{\tt CREATE} any system of unique identifiers would have
sufficed, but we chose the X window~ID for convenience.

\subsubsection{Window Management}

The first of the requests in this link of the system implement window
management functions.  The most obvious place to start is window
creation.  The request {\tt CREATE} sends the information on a newly
made X~window which the source editor is using to display the
particular buffer.  The window has already been created and mapped
and this request includes an implicit {\tt EXPOSE} of its entire
surface.

\begin{tabbing}
\hskip.5in\=\bf u\_short \hskip.5in\= Xoffset \hskip.5in\= X offset on window\\
    \>\bf u\_short \> Yoffset \> Y offset on window\\
    \>\bf u\_short \> width \> width of valid portion of window\\
    \>\bf u\_short \> height \> height of valid portion of window\\
    \>\bf u\_long \> fg\_pixel \> pixel value of foreground\\
    \>\bf u\_long \> fg\_pixmap \> tile for foreground\\
    \>\bf u\_long \> bg\_pixel \> pixel value of background\\
    \>\bf u\_long \> bg\_pixmap \> tile for background\\
    \>\bf u\_long \> hl\_pixel \> pixel value for high-light\\
    \>\bf u\_long \> hl\_pixmap \> tile for high-lighting\\
\end{tabbing}

The first four arguments define the region on the X window which is
owned by the proof editor.  The region (X, Y to X+width, Y+height)
is the only area on the X window which may be painted by the proof
editor.  The proof editor is not allowed to change the window in any
way other than repainting its designated region.  It may not even
use {\tt XClear} to erase the window since the title bar is drawn on
the same X window by the source editor.

The last six arguments are used as the X color handles (more~IDs) for
painting.  These are specified individually for each window since it
may be useful for different proof windows to have different colors
schemes.  The comments given after the element definitions should be
enough explanation to the X programmer.

The obvious accompaniment to create is {\tt DESTROY}, which is sent
with no trailing data.  A window may only be destroyed by the source
editor and no X access may be made to the window after the request
has been sent (and the editor once more reaches the top level).  All
proof editor data structures used for this window should be discarded
although cached page data should be expired by some other means for
efficiency when the user opens another window.

The two remaining functions for window management are {\tt RESIZE}
and {\tt EXPOSE}.  These both pass a rectangle although the rectangle
has different meanings in each.  The rectangle in {\tt RESIZE} is
just like that in {\tt CREATE}; it re-defines the drawing area on a
particular window.  For {\tt EXPOSE} the rectangle is offset from the
offset specified by the most recent {\tt RESIZE} or the {\tt CREATE}
(e.g., an {\tt EXPOSE} of the entire window would result in 0, 0 for
the exposed area's origin).  The area should always be within the
defined drawing area.

\subsubsection{Operation Batching}

A less important enhancement to the protocol is the notion of batching
requests.  The source editor will surround a set of requests which
form a single user command with a {\tt BSTART}/{\tt BEND} pair.  This
allows the proof editor to maintain state about flushing the X output
queue or improving the ``feel'' of the system by making the single
conceptual operation appear to occur all at once.

These batching groups may nest (for the sake of generality) although
groups deeper than one level will have the same significance as one.
The current source editor never uses more than one group at a time.

Each batch level applies to the window specified in the packet header
only, although in the current implementation no other messages will appear
during the batch grouping.

\subsubsection{Document Commands}

Each window has a notion of the current document (as well as page and
position) and the request {\tt DOCUMENT} requests that a different
document be displayed in the specified window.  The proof editor then
uses the document~ID given as the trailing data to query the formatter
for the page's formatted version.

The proof editor implicitly displays the document, displaying the
upper-left hand corner of the first (physical) page.  In the current
{\VorTeX}, the formatter can only handle one document, so any instance
of the proof editor will always be called with a single document~ID
for all windows.  As soon as the global document information
has been successfully gotten from the formatter, the proof editor must
send the source editor a {\tt DOCPAGES} request to inform it of the
number of formatted pages in the document.

The source editor requests to move between pages fall into two major
categories: physical and logical page specification.  Physical pages
are those established by the order in which {\TeX} writes pages into a
{\tt dvi} file.  To move between physical pages in a document, two
requests are used: {\tt GOTOABS} and {\tt GOTOREL}.  {\tt GOTOABS}
takes the physical page number and moves to that page, numbered from
one through the number of printed pages.  {\tt GOTOREL} just moves
forward the specified number or pages (negative numbers move backward).

Logical page specifications are defined by the numbers stored into the
ten {\tt \bsl count} registers.  The {\tt \bsl count0} register is used to
store the number to be printed at the bottom of the page ({\tt
\bsl pageno} is a synonym for {\tt \bsl count0}).  The other count registers
are used by various macro packages for other formatting counters.
Since there are ten count registers, a logical page specification may
have as many as 10 components, which match the numbers in the
{\tt\bsl count}s explicitly (as a number or range) or implicitly by being
omitted.  See the user document for the exact format of this page
specification (the source editor never needs to generate these
specifications itself--they always come from the user directly).
The {\tt LOGICAL} request passes a string to the proof editor for
interpretation.

The other layer of movement commands are those to move the page being
displayed in the window under the visible portion of the editor window.
There are two of these commands: {\tt MOVEABS} and {\tt MOVEREL}.
Both commands take an X and Y distance in pixels and cause motion
of the specified amount (for {\tt MOVEREL}, negative X and Y
mean left and up respectively).

There is one last positioning command, although currently it is the only
one of a one of a number of similar function.  The {\tt POSITION} request
specifies an {\nbox} in the document being displayed and merely instructs
the proof editor to ``make that box visible.''  This usually entails
moving to another page or moving on the current page, but is a very
unstructured command and is one of the major places where the proof
editor defines the feel of the system.

\subsubsection{Manipulating the Proof Selection}

``The proof selection'' is a concept which has a simple meaning to
the user and a complex one within the system.  In general, such a
selection is a list of \nbox es which happen to make up some logical
piece of structure on the page.  However, in the current implementation,
the current selection is a range of characters in the source buffer
which limits it to contiguous pieces of text which do not overlap
document structure (letters, words and paragraphs).

The selection is established (or re-established) with the {\tt SELECT}
request from the source editor.  The source editor specifies the
position of the mouse cursor and the proof editor must decide what
piece of the document this signifies.  To define a (new) selection,
the X and Y position of the mouse within the drawing area of the
window is sent.  To un-define the selection a {\tt SELECT} with no
trailing data is sent.

Once a selection has been defined, it may be expanded in two ways,
either by moving to a higher level of document structure or by
selecting a range of objects at the current level.  When the proof
editor receives a {\tt SELECT} at the same mouse position as the last,
it moves up a level (from a character to a word to a paragraph).  When
the mouse position changes, a new selection is made, starting again at
the character level.  Expanding a selection which involves more
structure at the same level as the current selection is done by
sending a {\tt SELECTMORE} request with a new mouse position.  All the
pieces of structure from the old selection through the unit defined by
the new mouse position become the current selection.

Whenever a selection is established or expanded, the proof editor must
return the first and last {\em character} in the selection using a
{\tt SELECTION} request, with the first and last \tbox~IDs as the
trailing data.  If there was an error, or the selection is undefined
for some reason, a {\tt SELECTION} request with no trailing data is
returned.

\subsection{Formatter/Source Protocol}

The Lisp functions which implement formatter commands are executed by
the source editor and result in remote procedure calls to the
formatter.  See the include file {\tt "ts\_comm.h"} for the exact
format of the formatter remote procedure calls.

In all formatter/source requests which require one, the {\em file~ID}
of the affected source file is passed in the {\tt commdata} field of
the message header (bytes 4 through 8).  The ID passed is constructed
by the source editor and can only range from 1 to 127.  These file~IDs
are assigned by the source editor at the time a {\TeX} buffer is first
dealt with and the file~ID is encoded into each character~ID it contains.

\subsubsection{Document Handling}

Currently, the {\VorTeX} formatter can only handle one document, so the
source editor closes down the formatter connection when a document is
closed and creates a new one when a new document is created.  Because of
this, there is no explicit request to create a new document.  The first
{\tt FORMAT} request contains an implicit document creation command.
The {\tt FORMAT} request contains the name of the master file (and the
corresponding file~ID in the packet header).  This establishes the
correspondence between the {\em master} or {\em root} file of the
document and its file~ID.

When the source editor creates a new document, it justs asks the formatter
to format the document.  The formatter then queries for the contents of
the file with a {\tt TEXINPUT} request, which specifies the file name in
the trailing data.  A {\tt TEXINPUT} is also sent when the formatter
encounters an {\tt \bsl input} command in the {\TeX} source.  Assuming the
file can be found, the source editor replies with a {\tt OPENFILE}
request which contains the contents of the source file.

The file~IDs and character~IDs within those files are assigned by the
source editor at the time the file is loaded into the system.  Each
character is represented by four bytes (the character~ID) internally.
This code contains the file~ID, the character code and a ``unique
number'' for that character code over all files.

In the present system, the source editor is the only piece that
directly changes the contents of the document.  This is done
(as the user edits a buffer) by sending {\tt INSERT} and {\tt DELETE}
requests to the formatter.  The {\tt DELETE} request just specifies
a range of characters (in the source buffer) to be deleted.  The
{\tt INSERT} request passes a position and a list of characters
to be inserted before that position.

\subsubsection{The Mapping Facility}

The source editor is constantly translating box~IDs and buffer/offset
values to perform the mapping functions.  These requests are implemented
as the only two cases of a more general facility, {\tt EXECUTE}.
An {\tt EXECUTE} request specifies a sub-request code and any trailing
data it requires and expects to receive a {\tt RETURN} reply.
The {\tt RETURN} contains the results of the {\tt EXECUTE} or no
data at all on error.  The two functions used for the mapping are
{\tt TGT2SRC} and {\tt SRC2TGT}.  The ``source'' representation of
a character is the file~ID and offset within that file.  The
``target'' representation is the \tbox~ID used in the formatter/proof
editor communications.\footnote{This is {\bf not} the same as the
character~ID assigned by the source editor.}

To implement the function ``scroll proof window to point,'' the source
editor queries the formatter for the {\tbox} which corresponds to the
text cursor in the source (using {\tt SRC2TGT}) and issues a
{\tt POSITION} command to the proof editor with the resulting box~ID.

To implement the proof selection in terms of the source editor model,
the first and last \tbox~IDs are translated into source buffer
positions (using {\tt TGT2SRC}).  Assuming the files are the same
and the positions in order, it can then assign the current point and
mark to the region defined by the proof selection.
