\section{Introduction}

{\VorTeX} is a source-based document preparation system (built on
Donald~Knuth's {\TeX} typesetting language\note{See {\it The \TeX book}
by Donald E. Knuth, Addison-Wesley Publishing Company.}) which allows
some of the functions of direct-manipulation systems as well.  This
report describes the editor portion of the prototype system as currently
implemented.

\subsection{Some Associations}

The system runs under X version 10, and appears to be another of the
{\EMACS}\note{??? EMACS ???} family of visual editors.  The command
structure and invocation is based on the {\EMACS} paradigm.  The
source editing paradigm is also that of {\EMACS} with point (the left
edge of the text cursor) and mark (an invisible text marker) used to
delimit text.

{\VorTeX} is actually implemented as several processes.  The one the
user starts up (which is called {\tt vortex}) implements the source
editor.  The source editor contains a Lisp interpreter which is used
to invoke the other processes, although this is typically done by the
system when needed.  From the user's point of view, he interacts
solely with the source editor since its Lisp interpreter performs all
interpretation and executes commands at the higher levels.

\subsection{A Few Words About the Editor}

The program {\tt vortex} is a text editor with a built-in Lisp interpreter
which has been written in C to perform all the user interaction for the
{\VorTeX} system.  It runs only under X, since is windows (in {\EMACS}
terminology) are also X windows.  Every different ``locus of editing''
is directed through a separate window.  In most instances, each window
views a different source file or page of the proof output, although it
is quite possible to edit the same file and view the same page at the
same time in different windows.

Our Lisp interpreter, vLisp, has been built into the editor (it kernel is
entirely written in C) and all the editor functions are accessed through
the Lisp system.  Higher level editing is done through vLisp functions.
The Lisp kernel and most primitive editing functions are coded in C and
almost all of the visible system is built up from these primitives through
vLisp functions.

\subsection{The Layout of \VorTeX}

\centerline{\psfig{figure=figs/system.ps}}
