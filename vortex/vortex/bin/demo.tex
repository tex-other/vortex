\magnification\magstep1
\def\VorTeX{V\kern-2.7pt\lower.5ex\hbox{O\kern-1.4pt R}\kern-2.6pt
  T\kern-.1667em\lower.5ex\hbox{E}\kern-.125emX}
\def\myitem#1#2{\bgroup
  \advance\rightskip by2\parindent
  \medskip
  \item{#1}{#2}\par
  \egroup}
\def\bul{$\bullet$}


\noindent \VorTeX\ makes complex document editing easier, because:

\myitem{\bul}{\VorTeX\ is incremental--only the changed pages need
to be reformatted (assuming the changes don't ripple to other pages).}

\myitem{\bul}{\VorTeX\ has a built-in displayer and editor--one can
point to the formatted output and even perform some editing on it.}

\myitem{\bul}{\VorTeX\ uses the \TeX\ language--it is not a toy
formatter.  \TeX\ can been used to format complex documents and
especially complex mathematics.}

\bigskip
Powerful personal workstatiosn with high-resolution displays, pointing
devices, and windowing environments have created many new
possibilities in presenting information, accessing data, and efficient
computing in general.  In the context of document preparation, this
workstation-based technology has made it possible for the user to
directly manipulate a document in its final form.  The central idea is
that a document is immediately reprocessed as it is edited; no
syntactic constructs are explicitly used to express the desired
operations.  This so-called {\it direct manipulation\/} approach
differs substantially from the traditional {\it source language
model\/}, in which document semantics (structures and appearances) are
specified with interspersed markup commands.  In the source language
model, a document is first prepared with a text editor, its formatting
and other related processors are then executed, usually in batch mode,
and the result is obtained.

The complete document development process involves a number of
subtasks ranging from authoring, reading, filing, to printing.  There
are certain aspects of document development that are best-suited to a
source-language approach while others are easier to deal with using
direct-manipulation techniques.  A hybrid paradigm combining the best
of both approaches seems most desirable.  In such a hybrid system, a
document has at least two representations: a {\it source\/}
representation with embedded commands that yields flexible high-level
abstractions, and a {\it target\/} representation displaying an
object's final appearance that gives precise placement and orientation
in response to direct manipulation.

Simultaneously maintaining more than one user-manipulable
representation of the same document is not an easy task.  In
particular, the historically batch-oriented processors that correspond
to source-to-target transformations would have to be made {\it
incremental\/}.  Furthermore, there must be a systematic way of
mapping changes from the target representation back to the source
representation.  Finally, an effective intermediate representation
needs to be derived in order to make transformations in both
directions possible.

\bye
