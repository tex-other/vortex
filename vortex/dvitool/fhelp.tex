% Copyright (c) 1986-1991 The Regents of the University of California.
% All rights reserved.
%
% Permission is hereby granted, without written agreement and without
% license or royalty fees, to use, copy, modify, and distribute this
% software and its documentation for any purpose, provided that the
% above copyright notice and the following two paragraphs appear in
% all copies of this software.
% 
% IN NO EVENT SHALL THE UNIVERSITY OF CALIFORNIA BE LIABLE TO ANY PARTY FOR
% DIRECT, INDIRECT, SPECIAL, INCIDENTAL, OR CONSEQUENTIAL DAMAGES, INCLUDING LOST PROFITS, ARISING OUT
% OF THE USE OF THIS SOFTWARE AND ITS DOCUMENTATION, EVEN IF THE UNIVERSITY OF
% CALIFORNIA HAS BEEN ADVISED OF THE POSSIBILITY OF SUCH DAMAGE.
%
% THE UNIVERSITY OF CALIFORNIA SPECIFICALLY DISCLAIMS ANY WARRANTIES,
% INCLUDING, BUT NOT LIMITED TO, THE IMPLIED WARRANTIES OF MERCHANTABILITY
% AND FITNESS FOR A PARTICULAR PURPOSE.  THE SOFTWARE PROVIDED HEREUNDER IS
% ON AN "AS IS" BASIS, AND THE UNIVERSITY OF CALIFORNIA HAS NO OBLIGATION TO
% PROVIDE MAINTENANCE, SUPPORT, UPDATES, ENHANCEMENTS, OR MODIFICATIONS.
%
% DOCSTR Sat Mar  4 16:43:58 1989

\input docmac.tex

\startdoc

\docstr{Prefix-1}
\desc{\em{Prefix-1} is a command to access an auxiliary keymap in the
Emacs tradition.  There are 3 such keymaps: the default keymap,
the \em{Prefix-1} keymap and the \em{Prefix-2} keymap.
\em{Prefix-1} is always bound to \lit{$<$ESC$>$}.
Normally, the user need not be aware of this command; it simply
exists more commands can be run from the keyboard.  You can
think of it as a different kind of shift key.}
\seea{{\it Prefix-2}}

\docstr{Prefix-2}
\desc{\em{Prefix-2} is the command to access the second auxiliary
keymap.  \em{Prefix-2} is always bound to \lit{$<$C-X$>$}.}
\seea{{\it Prefix-1}}

\docstr{ansi-keys}
\desc{This function interprets the codes sent by the Sun function keys
and attempts to do something intelligent with them.  The best 
solution would be to be able to map each of them to a function
just like any other key.  But since the function keys generate
strings of characters, it isn't easy to incorporate them into 
\lit{dvitool}'s key binding scheme.  So instead
\lit{dvitool} interprets each
sequence and runs a command from a private key map, much like
the key map the menu uses.  The person who installed \lit{dvitool}
can change the default bindings for the function keys; the
normal defaults are ``unbound'' for every key except the arrow
keys on the right keypad which scroll, L5 which runs 
\em{toggle-tool}, and L7 which runs \em{close-window}.
These binding are site-wide; once compiled in by the installer,
they cannot be changed by individual users.
This command is something of a hack; it was never intended to be
run intentionally by the user.  It's ability to interpret the
ANSI sequences depends on it being bound to \lit{\\e[} and you
will after you type some additional input, you will get innocuous
error messages if you run \em{ansi-keys} manually.  For all its
short comings though, it does suppress the unwanted characters 
that result when your fingers slip onto one of the function keys.}
\seea{{\it enable-ansi-keys}}

\docstr{ascii-of-selection}
\desc{This function displays a previously-made selection in ASCII
to make it easier to determine what the correct search string
for the selection would be.}
\seea{{\it erase-selection}, {\it extend-selection}, {\it select-char}}

\docstr{beginning-of-file}
\desc{This command positions \lit{dvitool} on the first page
of the \lit{DVI} file.}
\seea{{\it end-of-file}}

\docstr{bind-to-key}
\call{\bf function-name key-strokes}
\desc{This command ``binds'' a key (or keys)
to a function that will be executed when that
key is pressed.  The binding's life is the life of the 
\lit{dvitool} that executed it; to get bindings which take effect
on every instance of \lit{dvitool}, put the \em{bind-to-key}
command in your startup file.  Control and mouse characters
have ASCII representations which must be used in the startup
file.  \em{Describe-key} displays what command a particular
key sequence will invoke; as a side effect, it also displays
the ASCII representation of the sequence, so it can be 
used to easily determine the correct ASCII representation
for any valid input.}
\seea{{\it describe-key}, {\it dump-bindings}, {\it Startup-File}}

\docstr{bottom-edge}
\desc{This command positions the page image so the bottom edge of the
current \lit{DVI} page is visible.
If a numeric argument is given, the bottom edge of the
visible page (without the bottom margin) will be shown.}
\seea{{\it bottom-left}, {\it bottom-right}, {\it top-edge}}

\docstr{bottom-left}
\desc{This command positions the \lit{DVI} page so the bottom left
hand corner is visible.
If a numeric argument is given, the bottom left corner of the
visible page (without the margins) will be shown.}
\seea{{\it bottom-edge}, {\it bottom-right}, {\it top-left}}

\docstr{bottom-right}
\desc{This command positions the \lit{DVI} page so the bottom right
hand corner is visible.}
\seea{{\it bottom-edge}, {\it bottom-left}, {\it top-right}}

\docstr{bound-to}
\call{\bf command name}
\desc{This command describes all of the key strokes that will invoke
\sym{command name} separated by commas.}
\seea{{\it dump-bindings}}

\docstr{cd}
\call{\bf filename }
\desc{This command is similiar to the \em{sh(1)} or \em{csh(1)}
commands of the same name; it changes the directory.
When it is given a null argument, it changes to the user's
home directory.
The working directory can be viewed with the \em{print}
command; it can also be altered with the \em{set} command,
but since \em{cd} takes a file name which can be completed
on, and cwd is considered a string when changed with \em{set},
it is generally easier to use \em{cd}.}
\seea{{\it cwd}, {\it print}, {\it set}}

\docstr{close-window}
\desc{This command ``closes'' \lit{dvitool} to a small iconic shape.
The image that is painted and the position at which the image
is painted are controlled by user-definable variables.}
\seea{{\it init-icon-file}, {\it init-icon-x}, {\it init-icon-y}}

\docstr{describe-key}
\call{\bf key-strokes }
\desc{This command identifies what command a sequence of keystrokes
will invoke.  In the process it also echos the ASCII
character sequences \lit{dvitool} uses to represent all of 
the different input combinations.
This means that you don't have to remember that
control--shift--middle--mouse button is represented as
\lit{\\m\\\^M}; \em{describe-key} will tell you that.}
\seea{{\it bind-to-key}, {\it dump-bindings}}

\docstr{dump-bindings}
\desc{This command creates a file in the current directory
called \lit{dvitool.bindings} that 
describes which command each of the possible keyboard and mouse 
combinations will invoke.}
\seea{{\it bind-to-key}, {\it dump-commands}}

\docstr{dump-commands}
\desc{This command creates a file in the current directory
called \lit{dvitool.commands} that describes each of the 
\lit{dvitool} commands, the type of arguments that it takes,
and the key strokes it is bound to separated by commas.
In addition, all of the \lit{dvitool} variables and the their types
are described.}
\seea{{\it dump-bindings}}

\docstr{end-of-file}
\desc{This command positions \lit{dvitool} on the last page
of the \lit{DVI} file.}
\seea{{\it beginning-of-file}}

\docstr{erase-selection}
\desc{This command erases the current selection, if any.
It is primarily used to ensure that a search operation
begins at the top of the page.  A search begins
at the end of the current selection, if any, or the top
of the page, so erasing the current selection guarantees
that the search will begin at the top of the current page.}
\seea{{\it select-char}}

\docstr{exec}
\call{\bf command name}
\desc{\em{Exec} is the most general way to run a \lit{dvitool}
command.  \em{Exec} takes a (possibly completed) string
and looks in it's table of command names.  If some command
name exactly matches the string, then that command is 
executed.  Since the command name argument to \em{exec}
is subject to completion, a quick way to see all of
\lit{dvitool}'s commands is to type \lit{$<$ESC$>$x?}.}

\docstr{exit}
\desc{This is the command to use when you wish to end your session
with \lit{dvitool}.  It is usually bound to \lit{e}, 
\lit{$<$C-C$>$}, and \lit{$<$C-X$>$$<$C-C$>$}.}

\docstr{exit-help}
\desc{This command exits from the last instance of help to the
document you were viewing before you invoked help.  If the
stack of help entries is empty, an error message is printed.}
\seea{{\it help-commands}, {\it help-overview}, {\it help-variables}}

\docstr{expose-tool}
\desc{This command behaves exactly the same as the ``expose'' entry
on a the standard tool menu; i.e. it causes a partially covered
\lit{dvitool} to become the topmost tool, and thus unobscured.
If \lit{dvitool} is not covered by some other tool, this command
is a no-op.}
\seea{{\it hide-tool}, {\it toggle-tool}}

\docstr{extend-selection}
\desc{This command extends a previously made selection to include
new characters.  The command \em{select-char} only selects
1 character; thus, the usual method to selection a region
of characters is to first select one boundary character with
\em{select-char} and then to select the region by selecting
the other boundary character with \em{extend-selection}.
It is an error to run this command with no current selection.}
\seea{{\it ascii-of-selection}, {\it erase-selection}, {\it select-char}}

\docstr{find-file}
\call{\bf filename }
\desc{This command is the main entry point for viewing a \lit{DVI} file.
\em{Find-file} takes a filename, expands \lit{\~} and \lit{\$}
characters into home directories and environment variables 
respectively, appends \lit{.dvi} if necessary and attempts to read
the file and display it.  Various error messages are issued if
the named file is not a \lit{DVI} file or if it malformed etc.
The argument filename is subject to completion, so at any time
the user may enter a space character to attempt completion or
a question mark to see the list of choices \lit{dvitool} will
select from.  If a question mark or space character is needed
in the filename itself, precede it with a backslash.

In addition to completion, there is ``next file'' and 
``previous file'' selection, used in conjuction
with the \lit{DVI} files named on the command line.  Control-N
gets the next file name off of the list and control-P gets the
previous file name.  So to preview all of the chapters of
your upcoming book, invoke \lit{dvitool} like this:
\lit{\%\space dvitool\space ch*.dvi} and then use control-N and control-P
to select which file you'd like to view.  Just as with
space and question mark, if you really want a control-N or
a control-P in your filename, precede it with a backslash.}

\docstr{full-screen}
\desc{The \em{full-screen} command toggles the size of \lit{dvitool}
between as big as the physical screen will allow and whatever
size it was before \em{full-screen} was invoked.}
\seea{{\it zoom-horizontal}, {\it zoom-vertical}}

\docstr{goto-manuscript-page}
\call{\bf string }
\desc{This command searches for a specific page by examining
between 1 and 10 of {\TeX's} \lit{\\count} variables.
Each page of a \lit{DVI} file contains the value of these
10 variables when it was shipped out; with the proper macro
definitions, any number of section, chapter, or heading
configurations can be described with them.  The string that
\em{goto-manuscript-page} takes is parsed into up to 10 fields;
then each of the \lit{\\count} fields of each page is compared
with the parsed string until either a match is found or until
all the pages have been examined.  The format for the search
string is {$n[.n]\ldots$} where {$n$}
is either a decimal number
or the asterisk character \lit{*}.  The dot character is
a field separator.  Thus, assuming the the appropriate macros
have put the chapter number into \lit{\\count1}, the string
\lit{*.4} will find the first page of chapter 4 beyond the
current page.

\em{Goto-manuscript-page} also accepts strings of roman numerals
as input.}
\seea{{\it goto-physical-page}}

\docstr{goto-physical-page}
\call{\bf integer }
\desc{This command provides an alternative way to
\em{goto-manuscript-page} to seek to a \lit{DVI} page.  
Instead of the complexity of \lit{\\count} variables,
\em{goto-physical-page} simply displays the nth page
of the file.  Thus, \em{goto-physical-page}
with an argument of 1 is semantically equivalent to
\em{beginning-of-file}.}
\seea{{\it goto-manuscript-page}}

\docstr{help-commands}
\call{\bf help command name}
\desc{This command displays the \lit{dvitool} command help
file and then searchs for the optional argument in that
file.  The argument is subject to completion.  If completion
is used to generate the search string, a match is guaranteed.
The user may type in an arbitrary string as an argument.}
\seea{{\it help-overview}, {\it help-variables}}

\docstr{help-overview}
\call{\bf help overview string}
\desc{This command displays the overview file which describes
the general mechanisms of \lit{dvitool}.  The optional string
argument describes a search string to search for in the
overview file.  If completion is used to generate the string,
a match is guaranteed; however, the user may search for arbitrary
strings in the overview file.}
\seea{{\it help-commands}, {\it help-variables}}

\docstr{help-variables}
\call{\bf help variable name}
\desc{Like \em{help-commands}, this command displays the \lit{dvitool}
variable help file and then searchs for an optional string in
that file.  If completion is used to generate the search string,
then a match is guaranteed; the user may supply arbitrary strings
though.}
\seea{{\it help-commands}, {\it help-overview}}

\docstr{hide-tool}
\desc{This command behaves exactly the same as the ``Hide'' option
of the standard tool menu; i.e. it causes \lit{dvitool} to be
placed at the bottom of the stack of visible tools.}
\seea{{\it expose-tool}, {\it toggle-tool}}

\docstr{left-edge}
\desc{This command positions the page image so the left edge of the
page is visible.
If a numeric argument is given, the left edge of the
visible page (without the margin) will be shown.
This command is commonly used when one wants to see the
maximum amount of visible page.}
\seea{{\it bottom-left}, {\it right-edge}, {\it top-left}}

\docstr{list-all-commands}
\desc{This command simply displays the names of all of the commands.}
\seea{{\it dump-commands}, {\it list-all-variables}}

\docstr{list-all-variables}
\desc{This command simply displays the names of all of the variables.}
\seea{{\it dump-commands}, {\it list-all-commands}}

\docstr{magstep-0}
\desc{This function displays the \lit{DVI} file at its normal size.
It is a no-op unless the file has previous been displayed at
some other magstep.  Note that magnification in \lit{dvitool}
is global, that is, everything including the width of the page
and the margins will be affected by the magnification routines.
There is no analogy to {\TeX's} true points in \lit{dvitool}.}
\seea{{\it magstep-minus-5{$\ldots$}magstep-5}}

\docstr{magstep-1}
\desc{This command magnifies the document to 120 percent of its
\em{magstep-0} size.}
\seea{{\it magstep-minus-5{$\ldots$}magstep-5}}

\docstr{magstep-2}
\desc{This command magnifies the document to 144 percent of its
\em{magstep-0} size.}
\seea{{\it magstep-minus-5{$\ldots$}magstep-5}}

\docstr{magstep-3}
\desc{This command magnifies the document to 172.8 percent of its
\em{magstep-0} size.}
\seea{{\it magstep-minus-5{$\ldots$}magstep-5}}

\docstr{magstep-4}
\desc{This command magnifies the document to 207.4 percent of its
\em{magstep-0} size.}
\seea{{\it magstep-minus-5{$\ldots$}magstep-5}}

\docstr{magstep-5}
\desc{This command magnifies the document to 248.8 percent of its
\em{magstep-0} size.}
\seea{{\it magstep-minus-5{$\ldots$}magstep-5}}

\docstr{magstep-minus-1}
\desc{This command shrinks the document to 83.3 percent of its
\em{magstep-0} size.}
\seea{{\it magstep-minus-5{$\ldots$}magstep-5}}

\docstr{magstep-minus-2}
\desc{This command shrinks the document to 69.4 percent of its
\em{magstep-0} size.}
\seea{{\it magstep-minus-5{$\ldots$}magstep-5}}

\docstr{magstep-minus-3}
\desc{This command shrinks the document to 57.9 percent of its
\em{magstep-0} size.}
\seea{{\it magstep-minus-5{$\ldots$}magstep-5}}

\docstr{magstep-minus-4}
\desc{This command shrinks the document to 48.2 percent of its
\em{magstep-0} size.}
\seea{{\it magstep-minus-5{$\ldots$}magstep-5}}

\docstr{magstep-minus-5}
\desc{This command shrinks the document to 40.2 percent of its
\em{magstep-0} size.}
\seea{{\it magstep-minus-5{$\ldots$}magstep-5}}

\docstr{mouse-menus}
\desc{This command invokes the menus.  It may only be executed in
response to a mouse button, though \em{bind-to-key} will bind
it anywhere.}

\docstr{next-page}
\desc{This is the normal command to page through the \lit{DVI} file.
It displays the next page of the document with the upper left
hand corner of the page visible, easing the normal reading
flow from the bottom of one page to the top of the next.}
\seea{{\it goto-manuscript-page}, {\it next-page-positioned}, {\it previous-page}}

\docstr{next-page-positioned}
\desc{This command displays the next page of the \lit{DVI} file
at the same position on the page as the current page.  Thus,
if you are viewing the bottom of one page and want to see the 
bottom of the next page, this command suffices.}
\seea{{\it next-page}, {\it previous-page-positioned}}

\docstr{numeric-argument}
\desc{This command is used to enter a ``count'' which can be interpreted
by functions that take an integer as their first argument.
It is a prefix argument that is zeroed after each command.
The command interpreter checks each time a command is run for a
non-zero value of the numeric argument.  If it is non-zero and
the first argument of the command being executed is of type 
integer, the command interpreter substitutes the numeric argument
for the command argument.
For
example, \em{goto-physical-page} takes a single integer argument
and is usually bound to \lit{G}.  So the character sequence
\lit{1G} will position \lit{dvitool} on the first page of the
\lit{DVI} file.
Other commands use the presence or absence of a numeric argument
as a switch; for example, the edge commands act differently when
a numeric argument is present.}

\docstr{previous-page}
\desc{This is the previous analog to \em{next-page}.}
\seea{{\it next-page}, {\it previous-page-positioned}}

\docstr{previous-page-positioned}
\desc{This command is the previous analog to \em{next-page-positioned}.}
\seea{{\it next-page-positioned}, {\it previous-page}}

\docstr{print}
\call{\bf variable name}
\desc{This function shows the value of \lit{dvitool}'s internal
variables.}
\seea{{\it set}}

\docstr{redraw-tool}
\desc{This command behaves exactly the same as the ``Redisplay''
option of the standard tool menu; i.e. it causes \lit{dvitool}
to repaint all of its windows.}

\docstr{reload-fonts}
\desc{This function is identical to \em{reread-file} except that
before the file is reread, all of the characters in the
font cache are flushed.  This command is most useful when
some of the pixel images of characters in a \lit{DVI} file
have changed.}
\seea{{\it reread-file}}

\docstr{reread-file}
\desc{This command reloads and redisplays the current \lit{DVI}
file.  It is most useful when in the {edit--\TeX--preview}
cycle.}
\seea{{\it find-file}}

\docstr{right-edge}
\desc{This command positions the page image so the right edge of
the current \lit{DVI} page is visible.
If a numeric argument is given, the right edge of the
visible page (without the margin) will be shown.
This command is commonly used when one wants to see the
maximum amount of visible page.}
\seea{{\it bottom-right}, {\it left-edge}, {\it top-right}}

\docstr{scroll-absolute}
\call{\bf integer }
\desc{This command scrolls down vertically down the page by the amount
of the argument.  The argument is a percentage between 0 and 100
of the page to scroll down.}
\seea{{\it scroll-down}}

\docstr{scroll-down}
\desc{This command scrolls the page image down in the window if there
is any more of the page to display vertically.  The amount scrolled
by default is {$1/3$} of the vertical size of the window; 
the default
can be changed by supplying a numeric argument.  That argument
is taken as the denominator of the fraction of the window to scroll
down.  For example, the keystrokes \lit{10d} will scroll down
by {$1/10$} the vertical size of the window.}
\seea{{\it numeric-arg}, {\it scroll-left}, {\it scroll-right}, {\it scroll-up}}

\docstr{scroll-left}
\desc{This command is the leftward analog of \em{scroll-down}.}
\seea{{\it scroll-down}}

\docstr{scroll-right}
\desc{This command is the rightward analog of \em{scroll-down}.}
\seea{{\it scroll-down}}

\docstr{scroll-up}
\desc{This command is the upward analog of \em{scroll-down}.}
\seea{{\it scroll-down}}

\docstr{search-backward}
\call{\bf literal-string }
\desc{This command is exactly like \em{search-forward} except that 
it searches backwards through the document instead of forwards,
and the search is begun either from the last character of the
page, or the character before the first character of the selection.}
\seea{{\it search-backward-by-font}, {\it search-forward}}

\docstr{search-backward-by-font}
\call{\bf font-name literal-string}
\desc{This command is exactly like \em{search-backward} except that
a \lit{SET\_CHAR} must be in the named font to be a match.}
\seea{{\it search-backward}, {\it search-forward}}

\docstr{search-forward}
\call{\bf literal-string }
\desc{This command searches forward sequentially from a point on the 
current	page for a string of characters in the \lit{DVI} file.
The string is first parsed for any control sequences (any sequence
of characters beginning with the backslash (\lit{`\\'}) character.)
Then the search begins, either at the first character on the page
if there is no selection on the current page or at the first 
character after the selection.  Each character in the search
string is compared with a \lit{SET\_CHAR} operand in the \lit{DVI}
file or with a dvitool-created logical space character that
matches an ASCII space character.
The character matches if the integer parameter to the
\lit{SET\_CHAR} matches the integer value of the ASCII character.
If the end of the search string is reached, a match is reported.
Note that a match which spans a page boundary will not be found.
The variables \em{kern-threshold} and \em{line-break-threshold}
control how dvitool decides whether movements are interword
spaces or kerns.

Since a match is based on the position of the character in the
font, ligatures, in particular require some special handling.
The ``fi'' ligature for example is at position 12 decimal
in the roman font family.  This corresponds to the ASCII
character control-L.  So to search for the word ``file'',
you must use the string \lit{\\\^Lle}.  The command 
\em{ascii-of-selection} greatly eases the task of computing
the proper search string for a selection.

Searches can can be aborted prematurely by hitting the L1
key (usually the key at the upper-left most position on the
keyboard).  This is the same key that is used to reboot the
machine, but you needn't hold it down like a control key
to make it work; just depress it and release it.  
Every so often \lit{dvitool} will check to see if the key
has been depressed and will terminate the search if it has.
This key is effective only in the search functions.}
\seea{{\it ascii-of-selection}, {\it kern-threshold}, {\it line-break-threshold}, {\it search-backward}, {\it search-forward-by-font}}

\docstr{search-forward-by-font}
\call{\bf font-name literal-string}
\desc{This command is exactly like \em{search-forward} except that
a \lit{SET\_CHAR} must be in the named font to be a match.

\lit{Dvitool} keeps a cache of all the fonts you've used,
so the fonts presented as choices for the \em{font-name}
argument may or may not actually be in your DVI file.
This isn't a major concern, however, since \lit{dvitool} will
just issue an innocuous error message if you ask to restrict
searching to a font that doesn't exist in your current DVI file.}
\seea{{\it search-forward}}

\docstr{select-char}
\desc{This command makes a single character of the \lit{DVI} file the
current selection.  The character chosen is highlighted by 
displaying it in inverse video.  It may only be executed in
response to an input from the mouse.  To extend the selection
to more than one character, use \em{extend-selection}.
A ``double click'' will select a word (a sequence of contiguous
non-space characters).}
\seea{{\it ascii-of-selection}, {\it extend-selection}, {\it which-char}, {\it which-font}}

\docstr{set}
\call{\bf variable name  value string}
\desc{This command is used to change the value of \lit{dvitool}'s 
variables.  It can be run interactively, but is used most
often in the user's \lit{.dvitoolrc} file to customize 
\lit{dvitool}'s behavior.  This first argument is the name
of the variable to change and the second argument is the
value to change it to.  The second argument will of course
vary depending on the type of the variable to be changed;
to get the previous contents of the variable, type your
\lit{rprnt} character (usually control-R, see \em{stty(1)}).}
\seea{{\it print}}

\docstr{toggle-tool}
\desc{This command puts dvitool at the top of the windows visible
on the screen unless it already is the top window; in that case
it puts it on the bottom of the stack.}
\seea{{\it expose-tool}, {\it hide-tool}}

\docstr{top-edge}
\desc{This command positions the page image so the top edge of the
current \lit{DVI} page is visible.
If a numeric argument is given, the top edge of the
visible page (without the top margin) will be shown.}
\seea{{\it top-left}, {\it top-right}, {\it bottom-edge}}

\docstr{top-left}
\desc{This command positions the \lit{DVI} page so the top left
hand corner is visible.
If a numeric argument is given, the top left corner of the
visible page (without the margins) will be shown.}
\seea{{\it top-edge}, {\it top-right}, {\it bottom-left}}

\docstr{top-right}
\desc{This command positions the \lit{DVI} page so the top right
hand corner is visible.
If a numeric argument is given, the top right corner of the
visible page (without the margins) will be shown.}
\seea{{\it top-edge}, {\it top-left}, {\it bottom-right}}

\docstr{version}
\desc{Print a string which contains the version number of 
\lit{dvitool}.}

\docstr{which-char}
\desc{This command describes some of the properties of the first
character of the selection.  It reports the font name 
and the position of the character in the font in decimal,
octal, and hexadecimal.  If the selection contains more
than one character, it is trimmed down to the first charcter
in the selection and the information is reported on that
character.}
\seea{{\it ascii-of-selection}, {\it which-font}}

\docstr{which-font}
\desc{This command reports the font
name of the characters in the selection.  If there are characters from
more that one font in the selection, the selection is first trimmed to
include only characters from the same font as the first character in
the selection.}
\seea{{\it ascii-of-selection}, {\it which-char}}

\docstr{zoom-horizontal}
\desc{Make \lit{dvitool} as wide as the physical screen.  Like
\em{full-screen} and \em{zoom-vertical}, this command is a 
toggle; running it again will return \lit{dvitool} to its 
previous dimensions.}
\seea{{\it full-screen}, {\it zoom-vertical}}

\docstr{zoom-vertical}
\desc{This command offers the same functionality as clicking 
control middle button on the name stripe, e.g. the tool
becomes a tall as the physical screen while retaining
the current width.  This command is a toggle; running it
again will return \lit{dvitool} to its previous dimensions.}
\seea{{\it full-screen}, {\it zoom-horizontal}}

\enddoc
\bye
